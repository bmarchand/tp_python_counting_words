\documentclass{article}

\usepackage{mdframed}
\usepackage{xcolor}

\title{Discovering Python}
\author{Bertrand}

\begin{document}
\maketitle

\section{Basics: python scripts and print function}


A python script is a text file (with a \texttt{*.py} extension) containing lines of python code.
When launching in the terminal the command\footnote{the following command requires my\_script.py to be in the current directory.} \texttt{python3 my\_script.py}, the python interpreter
will read the script line by line and execute the lines of code, in order.
If the interpreter does not understand a line of code, it stops and returns an error in the terminal.

\paragraph{print function} In \texttt{my\_script.py}, there are 2 kinds of
lines. Some start by '\#': they are comments, and are ignored by the interpreter.
Others are of the form \texttt{print("something")}. \texttt{print} is a \textbf{function}, that prints out in the terminal whatever it is given within its
parenthesis (in this case, "something").  

\begin{mdframed}[backgroundcolor=black!20!white]
Try to execute the script \texttt{my\_script.py}\\
$\rightarrow$ Can you modify the script so that it prints \texttt{"Hello world"}
only one time ?
\end{mdframed}

\begin{mdframed}[backgroundcolor=green!30!white]
\textbf{Note}: The quotes (\texttt{""}) around \texttt{"hello world"}
are required, so that the interpreter understand it is given text that should
be printed directly (and not be read as python code).
\end{mdframed}

\section{Variables}

\subsection{Definition}

A variable is a piece of information (a number, a piece of text, $\dots$) that
we ask the program to \textbf{store} under a \textbf{name}.

For instance, if a line of code is \texttt{x = 10}, it means we are asking
the program to store the value 10 under the name \texttt{x}. We may then
use this variable as a number anywhere we want: to give to the print
function (\texttt{print(x)}), to define another variable (par exemple 
\texttt{y=x+2}) $\dots$

\begin{mdframed}[backgroundcolor=black!20!white]
Take a look at the script \texttt{variables.py}, and execute it. The questions
are in two parts:
\begin{itemize}
\item Part 1:
\begin{itemize}
\item[$\rightarrow$] What will be the value of d ? You can uncomment one  
of the lines to check.
\item[$\rightarrow$] Likewise, what will be the value of e ? As above,
print it to check. 
\end{itemize}
\item Part 2:
\begin{itemize}
\item[$\rightarrow$] Taking inspiration from the first note below, and
un-commenting 
the last line, how would you update \texttt{d} so that is value is augmented
by 2 ? 
\end{itemize}
\end{itemize}
\end{mdframed}

\begin{mdframed}[backgroundcolor=green!30!white]
\textbf{Note:} You can use the value of a variable to redefine it, and therefore
update it. For instance:
\begin{center}
\texttt{x = x + 3}
\end{center}
increments the value of variable \texttt{x} by 3. If it was 10, it is now 13.
\end{mdframed}

\subsection{Variable type}

Each variable has a \textbf{type}. For instance, the variables we have
just played with were either integers (for instance \texttt{d})
or strings of letters (for instance \texttt{s}).

\begin{mdframed}[backgroundcolor=black!20!white]
\begin{itemize}
\item[$\rightarrow$] Try to put the line \texttt{z = s+d} in \texttt{variables.py}, and execute it. What happens ?
Any idea why ?
\item[$\rightarrow$] The function \texttt{str()} allows to convert (pretty much) anything
into a string. Use it to convert d into a string so that the line above works.
Take inspiration from the notes below.
\end{itemize}
\end{mdframed}

\begin{mdframed}[backgroundcolor=green!30!white]
\textbf{Note:} Some functions give an \textbf{output} that you may store in a
variable, or use in the definition of a variable. For instance, the
function \texttt{str} outputs a string. \texttt{z = str(3)} is the 
same as \texttt{z = "3"}. If \texttt{a = 4} and \texttt{b = str(a)}
then \texttt{b} is the string \texttt{"4"}.
\end{mdframed}

\begin{mdframed}[backgroundcolor=green!30!white]
\textbf{Note: } The addition (\texttt{+}) on strings is the \textbf{concatenation}.
For instance, if \texttt{x="abc"}, \texttt{y="def"} and \texttt{z=x+y},
then \texttt{z} contains \texttt{"abcdef"}.
\end{mdframed}

\section{Loops and lists}

\subsection{Lists}

So far, we have seen two kinds of variables: integers and strings. We are
going to enrich our knowledge with \textbf{lists}. A list contains
several elements, along a given \textbf{order}.

For example, the line of code
\begin{center}
    {\tt l = ["a","b","c"]}
\end{center}

defines a list containing three strings, one equal to {\tt "a"}, one equal
to {\tt "b"}, and one equal to {\tt "c"}.

To access the elements of a list, one can use \textbf{indices}. For instance,
\begin{center}
    {\tt x = l[0]}
\end{center}
followed by {\tt print(x)} should print {\tt "a"}. This is because the first
element of a list is at position 0\footnote{Why not 1 ? It is more practical
for it to be 0 in more general contexts. \#shh \#trustme}. {\tt l[1]} would have
returned {\tt "b"}.

\begin{mdframed}[backgroundcolor=black!20!white]
\begin{itemize}
    \item $\rightarrow$ How can you minimally modify the code of {\tt lists.py}
        such that ``rouge'' is printed instead of ``bleu'' ?
\end{itemize}
\end{mdframed}

\section{Functions and methods}
\label{functions}

We have already encountered a two functions: {\tt print} and {\tt str},
which respectively print into the terminal, and convert variables to strings.

More generally, a \textbf{function} takes as \textbf{input} one or
several variables and either \textbf{outputs} other variables or
act on the variables it is given as inputs. Some functions are
\textbf{built-in}, that is to say already included in the python language.
It is the case of the {\tt print} and {\tt str} functions. A non-built-in
function is one the programmer defines. We are going to do that yet.

Examples of built-in functions are given in the following table
\begin{table}
\centering
    \begin{tabular}{|c|c|c|}
    \hline
    name & what it does & example \\
    \hline
        {\tt print} & print into the terminal & {\tt print("bonjour")}\\
    \hline
        {\tt str} & converts the input into a string & {\tt s=str(3)}\\
    \hline
        
    \end{tabular}
    \caption{Some built-in functions in python. This table is referred to in Section~\ref{functions}.}
\end{table}



\end{document}
