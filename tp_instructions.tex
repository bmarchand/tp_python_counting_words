\documentclass{article}

\usepackage{array}
\usepackage{listings}
\usepackage{mdframed}
\usepackage{xcolor}

\newcommand{\note}[1]{
\begin{mdframed}[backgroundcolor=green!30!white]
\textbf{Note:} #1
\end{mdframed}
}

\newcommand{\exercise}[1]{
\begin{mdframed}[backgroundcolor=black!20!white]
\textbf{Exercise:} #1
\end{mdframed}
}

\title{Discovering Python}
\author{Bertrand}

\begin{document}
\maketitle

\tableofcontents

\section{Basics: python scripts and print function}


A python script is a text file (with a \texttt{*.py} extension) containing lines of python code.
When launching in the terminal the command\footnote{the following command requires my\_script.py to be in the current directory.} \texttt{python3 my\_script.py}, the python interpreter
will read the script line by line and execute the lines of code, in order.
If the interpreter does not understand a line of code, it stops and returns an error in the terminal.

\paragraph{print function} In \texttt{my\_script.py}, there are 2 kinds of
lines. Some start by '\#': they are comments, and are ignored by the interpreter.
Others are of the form \texttt{print("something")}. \texttt{print} is a \textbf{function}, that prints out in the terminal whatever it is given within its
parenthesis (in this case, "something").  

\exercise{
Try to execute the script \texttt{my\_script.py}\\
$\rightarrow$ Can you modify the script so that it prints \texttt{"Hello world"}
only one time ?
}

\note{
The quotes (\texttt{""}) around \texttt{"hello world"}
are required, so that the interpreter understand it is given text that should
be printed directly (and not be read as python code).
}

\section{Variables}

\subsection{Definition}

A variable is a piece of information (a number, a piece of text, $\dots$) that
we ask the program to \textbf{store} under a \textbf{name}.

For instance, if a line of code is \texttt{x = 10}, it means we are asking
the program to store the value 10 under the name \texttt{x}. We may then
use this variable as a number anywhere we want: to give to the print
function (\texttt{print(x)}), to define another variable (par exemple 
\texttt{y=x+2}) $\dots$

\exercise{
Take a look at the script \texttt{variables.py}, and execute it. The questions
are in two parts:
\begin{itemize}
\item Part 1:
\begin{itemize}
\item[$\rightarrow$] What will be the value of d ? You can uncomment one  
of the lines to check.
\item[$\rightarrow$] Likewise, what will be the value of e ? As above,
print it to check. 
\end{itemize}
\item Part 2:
\begin{itemize}
\item[$\rightarrow$] Taking inspiration from the first note below, and
un-commenting 
the last line, how would you update \texttt{d} so that is value is augmented
by 2 ? 
\end{itemize}
\end{itemize}
}

\note{
You can use the value of a variable to redefine it, and therefore
update it. For instance:
\begin{center}
\texttt{x = x + 3}
\end{center}
increments the value of variable \texttt{x} by 3. If it was 10, it is now 13.
}

\subsection{Variable type}

Each variable has a \textbf{type}. For instance, the variables we have
just played with were either integers (for instance \texttt{d})
or strings of letters (for instance \texttt{s}).

\exercise{
\begin{itemize}
\item[$\rightarrow$] Try to put the line \texttt{z = s+d} in \texttt{variables.py}, and execute it. What happens ?
Any idea why ?
\item[$\rightarrow$] The function \texttt{str()} allows to convert (pretty much) anything
into a string. Use it to convert d into a string so that the line above works.
Take inspiration from the notes below.
\end{itemize}
}

\note{
\textbf{Note:} Some functions give an \textbf{output} that you may store in a
variable, or use in the definition of a variable. For instance, the
function \texttt{str} outputs a string. \texttt{z = str(3)} is the 
same as \texttt{z = "3"}. If \texttt{a = 4} and \texttt{b = str(a)}
then \texttt{b} is the string \texttt{"4"}.
}

\note{
\textbf{Note: } The addition (\texttt{+}) on strings is the \textbf{concatenation}.
For instance, if \texttt{x="abc"}, \texttt{y="def"} and \texttt{z=x+y},
then \texttt{z} contains \texttt{"abcdef"}.
}

\section{Loops and lists}

\subsection{Lists}

So far, we have seen two kinds of variables: integers and strings. We are
going to enrich our knowledge with \textbf{lists}. A list contains
several elements, along a given \textbf{order}.

For example, the line of code
\begin{center}
    {\tt l = ["a","b","c"]}
\end{center}

defines a list containing three strings, one equal to {\tt "a"}, one equal
to {\tt "b"}, and one equal to {\tt "c"}.

To access the elements of a list, one can use \textbf{indices}. For instance,
\begin{center}
    {\tt x = l[0]}
\end{center}
followed by {\tt print(x)} should print {\tt "a"}. This is because the first
element of a list is at position 0\footnote{Why not 1 ? It is more practical
for it to be 0 in more general contexts. \#shh \#trustme}. {\tt l[1]} would have
returned {\tt "b"}.

\begin{mdframed}[backgroundcolor=black!20!white]

\begin{itemize}
    \item $\rightarrow$ How can you minimally modify the code of {\tt lists.py}
        such that ``rouge'' is printed instead of ``bleu'' ?
    \item $\rightarrow$ Try out the following:
        \begin{lstlisting}[language=Python]
            for c in l:
                print(c)
        \end{lstlisting}
        Note that there must be exactly 4 spaces (or 1 tab\footnote{If it is
        not the case, the ``tab'' in the text editor in which you code should
        be re-defined as
        four spaces.}) before print(c).
        Can you explain what is happening ?
\end{itemize}
\end{mdframed}

\note{You have just seen your first example of a \textbf{for loop}.
Given a list (in the example above, {\tt l}), it goes through each elements
(in the example above, {\tt c}) and executes for each element the
same piece of code (in the example above, {\tt print(c)}). 
This piece of code must be tabulated four spaces further than the ``for'' keyword.
If it is not tabulated correctly, python will complain and not
execute the code.
}

\subsection{For loop}

Take a look at the file {\tt for\_loop.py}.
In its initial state, what it does is the following:
it iterates over each word in the list {\tt l} ({\tt l} is 
a list of strings, i.e.\,a list of words), and
prints the length of each word.
It then prints the variable {\tt total\_length}, but unless we modify
the code, this variable is just equal to 0.

\exercise{
\begin{itemize}
    \item $\rightarrow$ How can we modify the code so that
        at every iteration of the for loop, {\tt total\_length}
        is updated, in order to hold at the end of the loop the sum
        of the lengths of all words in the list ?
        Do not hesitate to take inspiration from the note below.
\end{itemize}
}


\note{Try out the following code:\\
{\tt x=0}\\
{\tt for y in [1,2,3]:}\\
{\tt \textcolor{green!30!white}{aaaa}x=x+y}\\
{\tt print(x)}
\\~\\
In this code, a for loop is run over the list [1,2,3] with variable {\tt y},
so {\tt y} takes value 1, then 2, then 3.\\
With this information, how do you explain the value that {\tt x} has acquired
at the moment
it is printed ?
}
\section{Putting it all together: counting words and characters in a text}

We have already encountered a two functions: {\tt print} and {\tt str},
which respectively print into the terminal, and convert variables to strings.

More generally, a \textbf{function} takes as \textbf{input} one or
several variables and either \textbf{outputs} other variables or
act on the variables it is given as inputs. Some functions are
\textbf{built-in}, that is to say already included in the python language.
It is the case of the {\tt print} and {\tt str} functions. A non-built-in
function is one the programmer defines. 
Examples of built-in functions are given in the following table.
\begin{center}
    \begin{tabular}{|c|c|c|}
    \hline
    name & what it does & example \\
    \hline
        {\tt print} & print into the terminal & {\tt print("bonjour")}\\
    \hline
        {\tt str} & converts the input into a string & {\tt s=str(3)}\\
    \hline
    \end{tabular}
\end{center}

Here, we will only use built-in functions. This section introduces
new built-in functions and built-in objects that, put together,
will allow us to count the number of words and characters in a
text given as input.

The following table introduces these new objects:

\begin{center}
\begin{tabular}{|c|c|m{7cm}|}
    \hline
    name & usage & what it does \\
    \hline
    {\tt open} & {\tt f = open(filename)} & Given the name of a text file (a string, in the example, called {\tt filename}), returns a {\tt File} (in the example, named {\tt f}). \\
    \hline
    {\tt readlines} & {\tt lines = f.readlines()} & This one is a \textbf{method},
    it is called on an object with a ``{\tt .}'' after the object. {\tt f} must be
    a {\tt File}, and the result, {\tt lines}, is a list of the lines of
    the text contained in the {\tt File} object. A line is whatever
    is between characters ``{\tt \textbackslash n}'', the end-of-line
    character. Each line
    is a string.\\
    \hline
    {\tt rstrip} & {\tt line = line.rstrip('\textbackslash n') } & Remove the end-of-line character from the end of a string, if it is present. \\ 
    \hline
    {\tt split} & {\tt words = line.split(' ')} & Given a string ({\tt line}),
    it ``cuts'' the string at every space ({\tt ' '}) and returns a list
    of the pieces of string in between the spaces ({\tt words} is a list of 
    strings).\\
    \hline
\end{tabular}
\end{center}

\exercise{
Given what you have learned on for loops,  variables, and printing, complete the
code in {\tt python\_script.py} so that at the end of the script, the total
number of words, and the total number of characters in the text ``text1'' are
printed. 
}


\end{document}
